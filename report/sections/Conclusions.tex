\newpage
\section{Conclusions}

\subsection{Performance}

The performances of the letter frequency job and letter count one show the impact of input size and number of reducers. Moreover, the comparison between the Combiner and In-Mapper Combiner implementations highlights the better performances obtained by using the \textit{In-Mapper Combining} design pattern.

\begin{itemize}
\item \textbf{Impact of Input Size}: larger input sizes, such as the 800 MB test file, grater the total processing time. This effect is evident across both the Combiner and In-Mapper Combiner implementations.
\item \textbf{Number of Reducers}: it is shown that increasing the number of reducers do not lead to an overall significant improvement. This is due to the overhead of managing multiple reducers and the limited amount of data to process. In most cases, the performance is optimal with just two reducers.

\item \textbf{Combiner vs In-Mapper Combiner}: the In-Mapper Combiner approach consistently outperforms the traditional Combiner method. The In-Mapper Combiner design pattern reduces the volume of intermediate data and minimizes data transfer overhead, leading to a more efficient MapReduce workflow.
\end{itemize}


\subsection{Qualitative Analysis}

The Letter Frequency Analysis provides significant insights into the variations and trends in letter usage across different texts and languages.

\begin{itemize}
\item \textbf{Letter Frequency Distribution (Figure \ref{fig:letter_freq_dist})}: the general letter frequency distribution graph highlights the overall trends in letter usage across the analyzed documents. Common letters like /e/, /a/, /t/, /i/, and /o/ show high frequencies, consistent with typical linguistic patterns in English and other Latin-based languages.

\item \textbf{Comparison Across Different Operas (Figure \ref{fig:letter_freq_dist_opera})}: when comparing letter frequency distributions across different operas, noticeable variations can be seen. These differences may reflect the unique stylistic and linguistic choices of the authors, as well as the historical and cultural contexts in which these operas were written. For instance, certain operas may exhibit a higher frequency of specific letters due to their thematic or lexical peculiarities.

\item \textbf{Italian vs. English (Figures \ref{fig:letter_freq_dist_lang}-\ref{fig:letter_freq_dist_lang_it})}: the comparison between Italian and English texts reveals distinct frequency patterns. Italian texts, for instance, tend to have higher frequencies of vowels such as /a/ and /o/, which are more prevalent in Italian phonology. Conversely, English texts show a higher occurrence of consonants like /t/ and /h/. These differences underscore the phonetic and orthographic characteristics inherent to each language.

\item \textbf{Top 5 Letters (Figure \ref{fig:top_5_letters})}: the analysis of the 5 most frequent letters across the documents provides further insights into common linguistic elements. In English, letters like /e/, /t/, and /a/ dominate due to their essential roles in constructing common words and grammatical structures. In Italian, vowels play a more prominent role, reflecting the language's phonetic structure.

\end{itemize}

\noindent In conclusion, the qualitative analysis of letter frequency data across different texts and languages highlights both universal and language-specific patterns. These findings not only reveal underlying linguistic trends but also enhance our understanding of how language and stylistic elements vary across different cultural and historical contexts.
